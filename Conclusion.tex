
\section{Conclusion and future work}
\label{sec:conclusion}

Hand-engineered, task-specific features are often task-specific and time-consuming to design. % Less adaptive than what? I changed this because there is no direct comparison
This difficulty is even more pronounced with multimodal data as we would like the features
to relate to multiple data sources.
%
In this paper, we presented a novel deep dynamic neural network (DDNN)
%that relied on  deep belief networks
%and 3D convolutional neural networks
for learning contextual frame-level representations and modelling emission probabilities in the framework of an HMM temporal model.
%
Different feature learning methods (DBN and 3DCNN) suited to the heterogeneous inputs from skeletal joints, RGB images, and depth images
were proposed, as well as different fusion schemes.
%
Experimental results on bi-modal gesture time series  show that the multimodal DDNN framework can learn
good models of the joint space of multiple sensory inputs, improving over  unimodal input.



There are several directions for future work.
%
Our results with those of other recent works suggest that learning features directly from data is a very important research direction
and that with more and more data and flops-free  computational power, learning-based methods are not only more generalisable to many domains,
but also are powerful in combination  with other well-studied probabilistic graphical models
for dynamical modelling and reasoning.
%
In this view, the learning of  better shared and complementary representation among multimodal  and heterogeneous  inputs,
as done in~\cite{neverova2014moddrop}, requires more exploration.
%
In addition, while the proposed HMM provided a good basis for the temporal modeling of gestures,
other more discriminant temporal approaches such as Conditional Random Field or further and better variants
\cite{wang2006hidden} could be directly exploited at their advantage in conjuction with our deep neural network  learning approach.
%
Ultimately, in a logical way, these two research directions converge into the investigation of
a single and unified deep learning framework fusing heterogeneous modalities
by using recent Recurrent Neural Networks such as Long Short Term Memory~\cite{graves2009novel} for modelling the temporal component of the problem.

%\mycomline{I would cite papers providing neural temporal models, rather than the one already applied to the gesture task}
%such as the work of ~\cite{DBLP:journals/corr/PigouODHD15}.

\endinput
