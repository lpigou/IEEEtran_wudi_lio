\begin{abstract}
This paper describes a novel method called Deep Dynamic Neural Networks \emph{(DDNN)} for  multimodal gesture recognition.
%
More precisely, a semi-supervised hierarchical dynamic framework based on a Hidden Markov Model (HMM) 
is proposed for simultaneous gesture segmentation and recognition where skeleton joint information, 
depth and RGB images are the multimodal input observations.
%
Unlike most traditional approaches which rely on the construction of complex handcrafted features, 
our approach learns high-level spatio-temporal representations using  deep neural networks suited to the input modality: 
a Gaussian-Bernouilli Deep Belief Network (\DBN)  to handle skeletal dynamics, 
and a 3D Convolutional Neural Network (\ThreeDCNN) to manage and fuse batches of  depth and RGB images.
%
This is achieved through the modeling and learning of the  emission probabilities of the HMM required to infer the gesture sequence.
%The framework can be easily extended by including an ergodic state to segment and recognize video sequences by a frame-to-frame mechanism, making online segmentation and recognition possible.
This purely data driven approach achieves a Jaccard index score of \textbf{\emph{0.81}} in the ChaLearn LAP gesture spotting challenge. 
The performance is on par with a variety of state-of-the-art hand-tuned feature-based approaches and other learning-based methods, therefore opening the door on using deep learning techniques to further explore multimodal time series.
\end{abstract}


\endinput








%% PREVIOUS VERSION
\begin{abstract}
This paper describes a novel method called deep Dynamic Neural Networks \emph{(DDNN)} for  multimodal gesture recognition.
A semi-supervised hierarchical dynamic framework is proposed for simultaneous gesture segmentation and recognition taking skeleton, depth and RGB images as input observations.
Unlike the traditional construction of complex handcrafted features, all inputs are learnt by deep neural networks: the skeletal module is modeled by Deep Belief Networks \emph{(DBN)}; the depth and RGB module are modeled by 3D Convolutional Neural Networks \emph{(3DCNN)} to extract high-level spatio-temporal features.
The learned representations are then used for estimating emission probabilities of the Hidden Markov Models to infer a gesture sequence.
The framework can be easily extended by including an ergodic state to segment and recognize video sequences by a frame-to-frame mechanism, making online segmentation and recognition possible.
This purely data driven approach achieves a score of \textbf{\emph{0.81}} in the ChaLearn LAP gesture spotting challenge. The performance is on par with a variety of the state-of-the-art hand-tuned feature approaches and other learning-based methods, opening the doors for using deep learning techniques to explore multimodal time series.
\end{abstract}
